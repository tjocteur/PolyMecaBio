\chapter*{Note d'intention}
\label{chapter:introduction}
\addcontentsline{toc}{chapter}{Note d'intention}
\markboth{Introduction}{Introduction}

\paragraph{}Ce polycopié est un support de cours pour une introduction à la mécanique destinée à de étudiant-es de licence 1 de biologie. La physique n'étant pas la spécialité de cette filière, ce sujet est traité de manière intermédiaire et laisse donc apparaître une rigueur assouplie. Toutefois, nous attacherons une importance à une définition précise des concepts afin que ces notions de mécanique puissent être réutilisées en dehors du chap biologique.

\paragraph{}Afin de dessiner un compromis avec l'inhomogénéité des formations à l'issue du baccalauréat, ce cours se veut d'un niveau intermédiaire à un cours de mécanique de niveau Terminale (enseignement de spécialité) et un niveau L1 en filière spécifique. Les étudiant-es ayant suivant l'enseignement de spécialité de physique-chimie en classe de terminale y trouveront donc bon nombre de rappels avec quelques nouveautés. Les étudiant-es ayant suivi l'enseignement de spécialité de mathématique devraient aussi être à l'aise avec les outils mathématiques utilisés et n'auront a priori aucune lacune déterminante pour suivre ce cours. Pour les étudiant-es n'ayant suivi aucun de ces deux enseignements, ce module de mécanique se révélera probablement plus exigeant et nécessitera sûrement parfois de combler quelques lacunes, essentiellement en termes de notions mathématiques. Même si le cours se veut accessible au maximum, il est fortement conseillée, si c'est votre cas, de reprendre des ouvrages de mathématiques niveau lycée pour vous mettre à jour sur les notions de vecteurs, dérivées et primitives.